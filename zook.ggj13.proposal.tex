\documentclass[]{article}
\usepackage{pdfpages}
\usepackage{hyperref}

%opening
\title{Game Design Processes in Rapid Game Development}
\author{Alexander Zook}

\begin{document}

\maketitle

\begin{abstract}
My proposed research will investigate the knowledge and creative processes involved in conceiving of a game and grounding an initial set of ideas in game mechanics. Particular details of this knowledge and process will be investigated using semi-structured interviews with completed games. Broader characterizations will be explored using a questionnaire distributed to the Global Game Jam participants. Combining in-depth interview information and large-scale survey information will provide insights into the creative practices involved in the game design process of value to researchers in many areas relating to the practices of game design, creativity in game jams, and the goals and tools of game jam participants.
\end{abstract}

\section{Research Objectives}
My research has been investigating the creative processes involved in game design and developing artificially intelligent (AI) tools to support these processes. The Global Game Jam (GGJ) presents an ideal avenue to examine the early creative ideation processes designers employ at the level of the knowledge they use and development process they go through when designing a game. In particular, my research objectives involve studying four primary areas:

\begin{enumerate}
\item Design goals---Characterizing the goals GGJ participants set for their games. Anecdotally the game design literature involves debates around the merit of different design goals (e.g. consider the breadth of approaches presented in game design reference sources \cite{fullerton2008:playcentric}\cite{salen2003:rulesplay} \cite{salen2006:reader} \cite{schell2008:gamedesign}). Design goals range from making a game fun for players \cite{koster2005:theory-fun} to creating immersion and a sense of flow \cite{fullerton2006:cloudgame} to conveying a meaningful message \cite{bogost2007:persuasive}. I aim to characterize this range of goals and understand how design goals are brought to bear on the design process.
\item Game ideation---Identifying the sources of creative inspiration designers draw from. Existing models of game ideation suggest there are many entry points for starting a design but tend to be based on theoretical analyses rather than existing practices \cite{hunicke2004:mda}. Example sources for game ideas include everyday activities a game is inspired by \cite{treanor2010:kaboom}, metaphors a game is meant to convey \cite{rusch2008:game-metaphor}, or models of systems \cite{crawford1984:gamedesign}. My goal is to characterize the kinds of knowledge GGJ game designers employ when defining the game they will create.
\item Game prototyping---Examining how designers ground inspirational ideas in particular game mechanics or systems. Initial inspirational ideas must be grounded in particular game systems and many designers report testing out a variety of potential ideas before settling on a system to employ \cite{gabler2005:7day-prototype} \cite{manker2011:prototyping} \cite{nelson2009:reqanal}. My goal is to characterize the process of mapping between ideas and mechanics designers employ.
\item Game refinement/balancing---Uncovering game balancing and refinement processes involved in terms of the features balanced and reasoning for balancing choices. Regardless of level of final ``polish'', game designs typically go through phases of some refinement of game systems to achieve the goals designers have set out \cite{schell2008:gamedesign}. While the GGJ puts tight time constraints on designers that limits the effort spent on this phase of design there are still many insights to be gained from what elements designers choose to refine, why and when.
\end{enumerate}

By studying participants at the GGJ I hope to gain detailed information on how game design processes occur, particularly focusing on creativity in realizing abstract game ideas in final designs. I hope to contribute to the body of knowledge of how game design occurs and use this knowledge to inform my subsequent development of AI tools to support game design creativity. 


\section{Research Process}
The research will involve a combination of semi-structured interview and questionnaires. Post-GGJ semi-structured interviews will be used to investigate the detailed processes involved in the four areas identified above. The interviews will involve an interviewed individual or team demonstrating a completed GGJ game and explaining the major decisions going into its design. I will hold interviews using Skype, Google Hangouts, or in person as is convenient to the participant. My questions (included below) will focus on the above four areas and will be used to ask designers about the detailed steps and developments during their game development. Designers will demonstrate their final game (and related prototypes, if applicable) and discuss the major design goals, inspirational sources, and prototyping and refinement processes involved. All individuals or teams I interview will be audio recorded for later transcription and review. Following the interview research participants will also be asked to complete a short questionnaire asking about demographic information.

To broaden the generality of my results and better understand the breadth of design approaches involved I will also distribute a questionnaire highlighting the primary questions involved in the interview process. While not providing the same level of detail, this broader survey will enable me to empirically examine the range of design approaches involved. Questions included (see below) cover the same topics as the interview process, emphasizing the four areas of design goals, game ideation, game prototyping, and game balance/refinement above.

I have already received prior approval from the Georgia Institute of Technology Institutional Review Board to perform post-design interview research, please see the attached approval documentation. My research does not involve on-site (i.e. during the GGJ) interviewing as I do not have existing approval for such work. Doing so would also preclude me from getting a breadth of data as I can only be present with one team at a time. 

%If possible within the constraints of the GGJ research proposal I would like to also perform on-site observation and audiotaping of a development team I observe (with prior participant consent) pending my approval for such research from the Georgia Institute of Technology Institutional Review Board. At this time I lack such approval but am currently seeking an amendment to my existing research approval to allow for on-site observation. My observation would consist of audio transcription, notes taken about the design process, and the same post-GGJ semi-structured interview and game walkthrough process as described above.


\section{Desired Help}
I would like help with two tasks for my research goals: (1) recruiting GGJ participants after the event to discuss their game and design process during the jam, and (2) distributing a survey addressing similar questions in a briefer format. Recruiting participants with a shared environment and experience will add robustness to my interview results by controlling for the amount of time allowed for the design and development process.

Distributing the questionnaire to the broader GGJ participant pool would complement the individual and team interviews. Wider survey results would provide evidence for the generalization of the interview results as well as covering the experiences of a wider range of participants. As with the surveys, the collection of this data from a large group of individuals participating in the same event will yield useful information on the four areas of my research that are comparable across people surveyed. 

In the sections below I provide details for the questions for the general GGJ participant survey (Section \ref{sec:survey_questions}) and specific interview and game walkthough questions for interview participants (Section \ref{sec:interview_questions}). Following these sections are my CV and ethics board approval for my research.

\section{Survey Questions}
\label{sec:survey_questions}

%\begin{itemize}
\begin{list}{\labelitemi}{\leftmargin=10pt \itemindent=0em \itemsep=0pt}
\item What was your initial goal for the game you made during the global game jam?
\item What inspirations or initial ideas did you have for your game? What was the starting inspirational source or goal for the game?
\item Why did you pick this particular idea for the game?
\item What problems did you encounter in developing your game?
\item What changes did you make to your initial idea as you worked on it during the game jam? Please describe the changes as small pieces of changes as possible.
\item What game mechanics and/or gameplay systems did you use in your game?
\item How did the mechanics or systems you made relate to the initial design ideas you had?
\item How did these mechanics change as you worked on the game during the game jam?
\item Did you prototype your game? If so, what kind of prototyping did you do and what did you learn from doing it?
\item What tools did you use to make your game? Please include software tools (e.g. programming languages or game engines) and any physical/analog materials (e.g. paper prototyping methods or storyboarding).
\end{list}
%\end{itemize}


\section{Interview Questions}
\label{sec:interview_questions}

The semi-structured interview use this initial set of questions:

\begin{itemize} \itemsep -2pt
%\begin{list}{\labelitemi}{\leftmargin=10pt \itemindent=0em \itemsep=0pt}
	\item What goals for the game did you have when you started?
	\item What inspirations did you have for this game?  (e.g. building off of an interesting mechanic from another game, trying to model a real-life process or experience, conveying a metaphor, etc.)? 
	\item Why did you pick those goals and ideas for the game?
	\item How did you start the design process? What did you first start doing?
	\item What game mechanics and/or gameplay systems did you use in your game?
	\item How did the mechanics or systems you made relate to the initial design ideas you had?
	\item How did these mechanics change as you worked on the game during the game jam?
	\item Did you go through prototyping with this game?
		\begin{itemize} \itemsep -2pt
		%\begin{list}{\labelitemi}{\leftmargin=10pt \itemindent=0em \itemsep=0pt}
		\item If so, what kinds of prototyping did you use?
		\item What did you learn from your prototyping and how did you use that information?
		\item What from your prototyping did you throw away?
		\end{itemize}
	%\end{list}
	\item How did you test or iterate on your design?
	\item In what ways did your final game differ from your original plan and expectations?
	\item What kinds of tools did you use for game design and development and why did you choose them?
	\item What were the key advantages and disadvantages of those tools?
	\item What obstacles or challenges did you encounter during the design process?
		\begin{itemize} \itemsep -2pt
		%\begin{list}{\labelitemi}{\leftmargin=10pt \itemindent=0em \itemsep=0pt}
		\item What were the major success and failures?
		\item What were you unable to do or had to work around?
		\end{itemize}
		%\end{list}
\end{itemize}
%\end{list}
The following questions are specifically targeted to the game walkthrough process:

\begin{itemize}
\item Where did the idea for [given game element] come from? Was [given game element] part of your initial inspirations? Prototyping? Developed through feedback?
\item Why did you choose to develop [given game element]?
\item Why did you choose to remove [given game element]?
\item How did [given game element] change over the course of development?
\end{itemize}



\bibliographystyle{abbrv}
\bibliography{lib}

%\pagebreak
%
%\includepdf[pages=-]{./azook_cv_working.pdf} \label{sec:CV}
%
%\pagebreak
%
%\includepdf[pages=-]{./IRB-approval_letter.pdf} \label{sec:IRB}


\end{document}
