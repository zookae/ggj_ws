% This is "sig-alternate.tex" V2.0 May 2012
% This file should be compiled with V2.5 of "sig-alternate.cls" May 2012
%
% This example file demonstrates the use of the 'sig-alternate.cls'
% V2.5 LaTeX2e document class file. It is for those submitting
% articles to ACM Conference Proceedings WHO DO NOT WISH TO
% STRICTLY ADHERE TO THE SIGS (PUBS-BOARD-ENDORSED) STYLE.
% The 'sig-alternate.cls' file will produce a similar-looking,
% albeit, 'tighter' paper resulting in, invariably, fewer pages.
%
% ----------------------------------------------------------------------------------------------------------------
% This .tex file (and associated .cls V2.5) produces:
%       1) The Permission Statement
%       2) The Conference (location) Info information
%       3) The Copyright Line with ACM data
%       4) NO page numbers
%
% as against the acm_proc_article-sp.cls file which
% DOES NOT produce 1) thru' 3) above.
%
% Using 'sig-alternate.cls' you have control, however, from within
% the source .tex file, over both the CopyrightYear
% (defaulted to 200X) and the ACM Copyright Data
% (defaulted to X-XXXXX-XX-X/XX/XX).
% e.g.
% \CopyrightYear{2007} will cause 2007 to appear in the copyright line.
% \crdata{0-12345-67-8/90/12} will cause 0-12345-67-8/90/12 to appear in the copyright line.
%
% ---------------------------------------------------------------------------------------------------------------
% This .tex source is an example which *does* use
% the .bib file (from which the .bbl file % is produced).
% REMEMBER HOWEVER: After having produced the .bbl file,
% and prior to final submission, you *NEED* to 'insert'
% your .bbl file into your source .tex file so as to provide
% ONE 'self-contained' source file.
%
% ================= IF YOU HAVE QUESTIONS =======================
% Questions regarding the SIGS styles, SIGS policies and
% procedures, Conferences etc. should be sent to
% Adrienne Griscti (griscti@acm.org)
%
% Technical questions _only_ to
% Gerald Murray (murray@hq.acm.org)
% ===============================================================
%
% For tracking purposes - this is V2.0 - May 2012


\documentclass{sig-alternate}

\usepackage{todonotes}
\usepackage{hyperref}

\begin{document}


% --- Author Metadata here ---
\conferenceinfo{Foundations of Digital Games Workshop on the Global Game Jam}{2013 Chania, Crete, Greece}
\CopyrightYear{2013}
%\conferenceinfo{WOODSTOCK}{'97 El Paso, Texas USA}
%\CopyrightYear{2007} % Allows default copyright year (20XX) to be over-ridden - IF NEED BE.
%\crdata{0-12345-67-8/90/01}  % Allows default copyright data (0-89791-88-6/97/05) to be over-ridden - IF NEED BE.
% --- End of Author Metadata ---

\title{Game Conceptualization and Development Processes in the Global Game Jam}
%\subtitle{}

\numberofauthors{1}
\author{}
%\author{
%\alignauthor
%Alexander Zook and Mark O. Riedl\\
%       \affaddr{School of Interactive Computing, College of Computing}\\
%       \affaddr{Georgia Institute of Technology}\\
%       \affaddr{Atlanta, Georgia, USA}\\
%       \email{\{a.zook, riedl\}@gatech.edu}
%}

\maketitle
\begin{abstract}
\todo[inline]{update to results}
My proposed research will investigate the knowledge and creative processes involved in conceiving of a game and grounding an initial set of ideas in game mechanics. Particular details of this knowledge and process will be investigated using semi-structured interviews with completed games. Broader characterizations will be explored using a questionnaire distributed to the Global Game Jam participants. Combining in-depth interview information and large-scale survey information will provide insights into the creative practices involved in the game design process of value to researchers in many areas relating to the practices of game design, creativity in game jams, and the goals and tools of game jam participants.
\end{abstract}

\todo[inline]{decide category, terms, etc.}
% A category with the (minimum) three required fields
\category{H.4}{Information Systems Applications}{Miscellaneous}
%A category including the fourth, optional field follows...
\category{D.2.8}{Software Engineering}{Metrics}[complexity measures, performance measures]
\terms{Theory}
\keywords{ACM proceedings, \LaTeX, text tagging}



\section{Introduction}
% % GGJ gives unique opportunities and challenges to study process of rapidly creating a game from a rough inspiration
The Global Game Jam (GGJ) provides a unique opportunity to study the process of conceiving and building a game de novo within tight time constraints. Strict time limits enable the study of the game design and development process at a level of detail normally not possible, with massive participation (roughy 16,700 participants) allowing for large-scale patterns and analysis. 
However, these opportunities come with methodology challenges for studying the design process. What are effective methods for understanding practices that can balance the scale of the GGJ with rigorous, detailed analysis? How can the unique structure of the jam be accounted for to help generalize results from GGJ participants to broader game design practices and methods?


% % enables unique lens on compressed process of conceptualizing and realizing design ideas - good opportunity to study process
In this paper we study the compressed development process of conceptualizing a game and realizing the game in a working product at the 2013 GGJ. Studying this process is challenging---building a rigorous theory of the time-limited development process requires understanding how designers choose ideas and develop their game ideas, and how this relates to the dynamics of group collaboration and code-level implementation. 
We take a multi-step approach: 
using survey instruments to first characterize the space of game design process
and then follow with more detailed studies of aspects of these processes.
This paper describes the results of a free-response survey we administered to GGJ participants about their design inspirations and goals, process of implementing those ideas in a game over the course of the GGJ, how they refined their game, and pitfalls they encountered along the way. 
We find common trends in inspirations for game ideas, design goals for games, and process for implementing a design into a working game.
%, and common issues encountered in developing the ideas. 
We conclude with a discussion of ways to deepen this analysis through triangulation with other research methods and some of the implication of our results.


% % studying is challenging - to build a deep theory of the process requires understanding the space of how designers go about deciding on ideas and developing them, and how this relates to process of collaborating and implementing ideas

% % first study in overall process: overview in survey questions, then follow-ups planned to get detailed information on process. at least two important approaches: (1) semi-structured interviews to get more in-depth information on what occurred and (2) protocol analysis for detailed information on the process, representations, and flow among them.

% % emphasis on process of ideating game and realizing: inspirations and goals for design, flow of process of developing ideas into concrete code (prototyping + realizing + refining), pitfalls along the way that highlight major challenges in process.

% % close with limitations of the study and implications for how to move forward

\section{Related Work / Study Topics}

We examine the process of conceptualizing and realizing a game, specifically: inspirations and goals for a design, and the process of developing ideas and turning them into concrete running code (prototyping, implementing, and refining).
%, and pitfalls GGJ participants frequently encountered.

The GGJ emphasizes values of experimentation and innovation and we sought to characterizing the design goals and insirational sources GGJ participants set for their games. Anecdotally, the game design literature had debated the merit of different design goals (e.g. consider prominent game design texts \cite{fullerton2008:playcentric}\cite{salen2003:rulesplay} \cite{salen2006:reader} \cite{schell2008:gamedesign}). Design goals range from making a game fun for players \cite{koster2005:theory-fun} to creating immersion and a sense of flow \cite{fullerton2006:cloudgame} to conveying a meaningful message \cite{bogost2007:persuasive}. Bogost \cite{bogost2011:howto} catalogs a plethora of uses for games---from inducing relaxation to drilling skills---using existing game examples. Despite this rich discussion, little empirical work has examined the space of design goals. We examine the range of design goals GGJ participants set for their designs.

Game inspirations are the sources of ideas designers draw from when conceptualizing their game. Existing models of game conceptualization suggest there are many entry points for starting a design, but tend to be based on theoretical analyses rather than existing practices \cite{hunicke2004:mda}. Example sources for game ideas include everyday activities a game is inspired by \cite{treanor2010:kaboom}, metaphors a game is meant to convey \cite{rusch2008:game-metaphor}, or models of systems \cite{crawford1984:gamedesign}. We empirically examine the kinds of inspiration GGJ participants draw on and discuss their relation to the use of themes in the GGJ.

Developing an idea into game mechanics involves grounding the abstract game concept into a set mechanics through: (1) prototyping designs, (2) implementing running code, and (3) refining the game. Throughout this process there is potentially a feed loop between the game artifact and conceptualization. Initial inspirational ideas must be grounded in particular game systems and designers vary in how they approach the problem \cite{gabler2005:7day-prototype} \cite{manker2011:prototyping} \cite{nelson2009:reqanal}. Some approaches emphasize iterative playtesting \cite{fullerton2008:playcentric} \cite{schell2008:gamedesign} while others test a breadth of small ideas before settling on an idea \cite{gabler2005:7day-prototype}. Prototyping may leverage paper models \cite{manker2011:prototyping}, abstract models \cite{nelson2009:reqanal} \cite{dormans2011:machinations2}, or simple code \cite{gabler2005:7day-prototype}. We examine the use of prototyping in the GGJ and approaches designers take to realizing their ideas in running game code.

Game balancing and refining processes involve finding aspects of the game that should be altered, selecting among those aspects, and choosing how to change them. Regardless of level of final ``polish'', game designs typically go through phases of some refinement of game systems to achieve the goals designers have set out \cite{schell2008:gamedesign} \cite{fullerton2008:playcentric}. We examine what aspects of GGJ games were altered and how these changes were made.

\section{Method}
We provided a ten question open-response survey that was administered online as part of the post-GGJ extended survey (Appendix \ref{sec:survey}). We gathered and manually coded responses to each of the questions into categories, allowing multiple possible codes for responses.


\section{Results}
Of approximately 16,700 participants in the GGJ, 420 provided responses to at least one of these questions.
Below we discuss broad categories of responses within each of the study topics: inspirations, design goals, prototyping methods, and the flow of realizing game ideas in code.

%General note: game jam theme has strong influence on participant results. Many used to ground ideation process. Jam constraints focused on development problems, with playtesting and prototyping often being cut. Team management is not frequently mentioned.

\subsection{Inspirations}
Participants drew from a breadth of sources for inspiration, including: other games, abstract concepts, emotions, life experiences, art styles, biological systems, books and poems, music, and films. Many mentioned explicit use of the GGJ theme---the sound of a heart beating in 2013---inspiring the use of rhythm in game mechanics, biological hearts as model systems, and life experiences of love. Interestingly, some intentionally sought to have more distant connections to the theme:
``QUOTE'' \todo[inline]{quote text} 

Drawing from life experiences \cite{anthropy2012:zinesters}, systems to model \cite{crawford1984:gamedesign}, and other media \cite{bogost2011:howto} are all recognized in the game design literature as important sources of inspiration.
We found life experiences references in terms of specific memories (e.g. watching a blind-friendly TV show) as well as general activities (e.g. holding a conversation).
Biological systems---particularly the heart and associated diseases---were a source for some designs, primarily due to the heart beat jam theme.
Non-game media provided initial grounding through scenarios (e.g. Edgar Allan Poe's ``The Telltale Heart'') or more general characters and concepts (e.g. the Borg from ``Star Trek'').

GGJ participants also tapped their gaming knowledge, in terms of specific target mechanics or reference games.
The heart beat theme lead many projects to emphasize rhythm mechanics or incorporate rhythm elements into another genre. In general, mechanics proved to be a frequent starting point, representing the most commonly cite category \todo[inline]{number} after the theme \todo[inline]{number}.
Game references included specific digital games (e.g. Super Mario Bros.), playground or field games (e.g. Simon Says or tag), tabletop games (e.g. \todo[inline]{find example}), or broad game genres (e.g. platformer, card game). Game references were connected to game mechanics, art styles, controls, ``feel'', etc. Overall, game references targeted single-player games and action-oriented genres (side-scrolling runner, platformer, one-button games, etc.).

Together, these results demonstrate the GGJ community matches many professional game design norms, skewed by the theme and limitations of the GGJ. The GGJ theme pushed many participants on more abstract concepts than often seen in industry staple titles focused on fantasy adventures and science fiction. Strict time and resource limits focused development on game genres that are quick and easy to implement and refine. Together, these dual tensions make the GGJ a powerful incubator for small games that experiment with new themes and mechanics. However, this likely limits the value of the GGJ in training novices to build more complex game genres---real-time or turn-based strategy, role-playing games, simulation games, etc. Future GGJ efforts may target ways to support experimentation in these other avenues through alternative support structures, durations, or means for participating.

\subsection{Design Goals}
Three broad categories of goals seem to drive GGJ participants: (1) system level goals, (2) personal goals, and (3) player-oriented goals.
\todo[inline]{examples}
System-level goals emphasized creating a game of a certain type or that met certain design criteria. Participants emphasized recreating other games, trying out new mechanics, having an original game, or attempting to convey a theme through the game structure. The GGJ theme and emphasis on innovation inspired some to set a goal for the final game system and strive to realize the conceived system in a concrete, running game.

Personal goals focused on benefits to GGJ participant themselves. 
The single most common goal \todo[inline]{number} was to make and finish a game. 
Many sought to learn skills (particularly programming, but also art, audio, and design), network with others, build their portfolio, test potential ideas for later expansion, enjoy the game creation process, or even ``win the competition.'' Participants see the jam as an opportunity to test out game development or seed their future projects. An emphasis on competition among some is particularly interesting given the GGJ site explicitly states: ``The GGJ is not a competition.''\footnote{\url{http://globalgamejam.org/about}}

Player-oriented goals emphasize the person(s) engaging with a game. Fullerton et al. \cite{fullerton2008:playcentric} emphasize a play-centric approach to design focusing on players while McGonigal \cite{mcgonigal2011:realitybroken} highlights the potential for games to have individual and societal impacts.
GGJ participants referenced goals of players enjoying the game, learning about a new topic (e.g. bee colony collapse disorder), or engaging in critical thinking about a topic. Societal-level design goals aimed to raise awareness about world issues or even ``change the world.''

Compared to the standard design mantra of focusing on player experience, the GGJ encourages a broader range of goals for personal gain, social improvement, or innovation. 

%player-centered design de-emphasized, likely due to jam constraints. suggests GGJ currently emphasizes a process for game development rather than game design.

\subsection{Prototyping and Development Processes}
The notion of prototyping was complicated by the strict duration of the game jam. Many noted that their either was no time to prototype or that they considered their final game a prototype in itself. Others described a process that began as prototyping, but evolved into the final game itself.

Prototyping processes broadly employed either paper prototyping or engine prototyping. Paper prototyping used whiteboards, paper drawings, or various tokens and pieces to simulate game systems and mechanics before beginning to code the game. Relatively few participants mentioned the use of paper prototypes, possibly due to lack of experience and familiarity or the limitations of the jam. However, those that did paper prototype described it as a beneficial practice:
``\todo[inline]{prototyping quote?}''

Engine prototyping approaches varied along a spectrum ranging from parallel testing and development of mechanics in isolation to serial additions of separate mechanics, each tested alone before being integrated. Mechanics, levels, characters, physics, controls, animations, movement, and user interfaces were all subject to this prototyping process. 
Participants often reported developing initial prototypes in game creation software (most commonly Unity) with the intent to switch to a more complex development, only for the initial prototype to evolve into the final game. In these cases features would incrementally be added to the core game until the end of the jam.

Many set out to intentionally pursue an iterative process of building up from a simple target game. Unlike the above case, these descriptions emphasized the goal of getting a game working and then using testing and feedback to iteratively add complexity and features as needed.

Parallel development approaches deliberately built and tested features for the game in isolation before combining them into the final product. Many different systems were tested alone before being combined to produce a working game. In contrast to serial approaches, these efforts focused on ensuring all the desired systems were functional before being combined and parallelized work among team members.

Compared to professionally described best practices \cite{fullerton2008:playcentric} GGJ participants as a whole describe relatively little use of player feedback. Most development focused on becoming familiar with tools to implement mechanics and removing obvious bugs that inhibited game functionality. Our question on problems encountered mentioned programming and acquiring and using game development tools as the most pressing challenges. Given the GGJ is focused on providing programming experience this may not necessarily be problematic, although future jams may consider alternative ways to prepare or support learning basic game creation tools before or during the jam to alleviate some of these problems.

\subsection{Realization}
Realizing an initial game concept as an implemented game typically involved changes both to the initial ideas from final game. Responses to questions about changes to the game mechanics and game ideas were often interchangeable and we report on their combined results here.

GGJ participants managed the set of game features and game artifact in three ways:
(1) starting from many ideas and iteratively reducing scope based on dependencies or feasibility;
(2) starting from vague ideas and building out mechanics and ideas during implementation;
and
(3) starting from a core idea and building it up based on testing and feedback.
Shifting ideas were largely implemented using the same broad sets of tactics: adding or removing whole mechanics, swapping out one mechanic for an (often simpler) alternative, and fine-tuning and balancing a mechanic. Some participants also included details on changes to the game objectives, background story and theme, art assets and animations, or functionality of multiplayer interactions.

Scope reduction occurred by cutting out intended features, taking complex or large mechanics and systems and reducing the number of elements involved, or substituting more complex systems for simpler alternatives.
Cutting features served to reduce the overall functionality of the game, typically because the magnitude of the task to implement them was infeasible or time simply ran out. Participants would remove both systems within the game (e.g. attacks requiring combinations of buttons rather than single buttons) or reduce the total number of components used (e.g. fewer game levels or types of enemies).
Reducing mechanics more often occurred when already implemented systems were buggy or dysfunctional or when playtesting (personally or with others) showed them to be overly complex or unintuitive. 
Iterative scope reduction was the most common way participants described their process and was typically due to development constraints.

An alternative approach involved only vaguely specified the game systems before attempting to build an initial game. Once that game was built further mechanics and ideas would be added by riffing off of the core systems and crystallizing the general idea.
\todo[inline]{example}
Rather than carefully plan out a full game, a vague inspiration would seed the process of solidifying ideas through iterative feedback with their realization.
In some cases this lead to an unintentional change in game genre due to incremental shifts in mechanics or a shift in design goals:
``\todo[inline]{quote about changing}''

Participants also described player-centered approaches that explicitly planned a small game and relied on testing and external feedback to build off that core game. Unlike starting from a vague idea these games were described as built from a process of trying to get a product to players rapidly in order to refine and extend it as needed. These approaches put heavier emphasis on a polished final product, rather than realizing an abstract idea.

Overall, the GGJ seems to serve as an important venue for developing experience in programming and realizing a final game based on a personal vision. The jam time constraints (and lack of available testers) generally steer development practices toward personal responses to the product, rather than player testing and feedback. These results open questions as to whether the GGJ can more effectively encourage standard industry practices such as paper prototyping and player testing and whether these industry methods are amenable to innovative practices under strict time constraints.

%Aside mechanics, visualization and controls were occasionally mentioned as aspects developers would change along the way. Balance and feature polish (particularly art) were also discussed, but were less central aspects. The GGJ focuses on integrating many parts over learning how each task works. heavy programming challenges suggest biggest issue and focus on learning is around how to get code up more than exercising other skills.

%idea changes most frequently driven by development needs rather than playtest results. testing primarily seems to have been based on personal reactions to product rather than player feedback. GGJ seems to encourage ``personal'' games that meet one's own standards more than play-centric designs. 
\todo[inline]{cite other study of professional designer types?}



\section{Methodological Limitations and Implications}
% % note limitations of survey method: wasn't able to get great detail on flow of process; focus on high-level changes; only those who opt-in
The voluntary survey methodology we has important limitations in coverage of GGJ participants and the depth of response data gathered. Only 420 of 16,700 participants responded to these questions. Respondents were likely skewed toward successful projects and those more invested in the GGJ. Thus, we cannot easily examine similarities or differences in design processes between those who successfully complete the GGJ and those who do not. Future research will require better methods to automate survey administration and collection or ensure randomized sampling from participants.

Survey responses are limited to the most salient aspects of an experience, limiting the level of detailed processual information gathered. We cannot make strong conclusions about the cognitive or social processes involved in game development from this form of data. Retrospective protocol analysis---where participants are recorded and asked to then view this recording and narrate their thinking---is one means to gather such detailed data, although constrained to a smaller scale than we employed. As retrospective protocols are typically used for short sessions (up to hours) they may require modifications to find and examine only key points in the process, or to use a ``fast-forward'' viewing approach. 

Semi-structured interviews provide another alternative to gathering detailed data. Interviews are limited by being primarily subjective data, but require far less time and detailed data than protocol analysis while still gaining useful qualitative insights. Using prompt materials gathered over the course of the GGJ---such as in-process game builds from source control, photos or video of onsite activity, and observer notes on the development process---may amerliorate some of the biases around memory salience.

Our survey did not identifying information on participants. Understanding how team members differ in their thinking and process was not possible as a result, preventing a more in-depth study of how collaboration impacts the conceptualization and development process at the GGJ. Employing a retrospective protocol or semi-structured interview with individuals and subsequently groups is one means to collect such information.


\section{Conclusions}
GGJ goals: innovation, experimentation, collaboration, creativity
- innovation: inspirations are relatively narrow range
	- some came to innovate, but most came to make things
	- note that many drew inspirations that came from either side of interpreting "heart" - often macabre responses
- experimentation: many are involved to simply make a game
	- strong inspirations from existing games and desire to make basic game
	- Q: what would better foster experimentation?
- collaboration: few interested explicitly in networking or working with team; often complain about team problems
	- Q: are there better ways to foster collaboration through matchmaking methods or systems? pre-jam pairing or at-jam?
- biggest problems are around programming + collaborating
	- mismatch between what participants seek (short duration event to make game) and what jam provides

More research into differences in outcomes based on processes for approaching: pruning vs expanding. In-depth protocol analysis to get detailed evolution. Comparison of design methods by levels of experience.

Alternate formats of jam for skill sets: playtest-focused, full game development, etc.

need better ways to understand how constraints impact results, particularly when aspects like time limits vary. need better ways to automatically record many aspects: participant outcomes and experiences, products, practices employed over development process.

clearly provides useful guidance on priorities and perspectives. open question responses helped address a breadth of issues without guiding participants to only answer key points.

%ACKNOWLEDGMENTS are optional
\section{Acknowledgments}
Global Game Jam committee. Global Game Jam participants and survey respondents.


% The following two commands are all you need in the
% initial runs of your .tex file to
% produce the bibliography for the citations in your paper.
\bibliographystyle{abbrv}
\bibliography{lib}


% ACM needs 'a single self-contained file'!

\pagebreak

%APPENDICES are optional
%\balancecolumns
\appendix
%\todo[inline]{possibly use for questions given}
\section{Survey Questions}
\label{sec:survey}
\begin{list}{\labelitemi}{\leftmargin=10pt \itemindent=0em \itemsep=0pt}
\item What was your initial goal for the game you made during the global game jam?
\item What inspirations or initial ideas did you have for your game? What was the starting inspirational source or goal for the game?
\item Why did you pick this particular idea for the game?
\item What problems did you encounter in developing your game?
\item What changes did you make to your initial idea as you worked on it during the game jam? Please describe the changes as small pieces of changes as possible.
\item What game mechanics and/or gameplay systems did you use in your game?
\item How did the mechanics or systems you made relate to the initial design ideas you had?
\item How did these mechanics change as you worked on the game during the game jam?
\item Did you prototype your game? If so, what kind of prototyping did you do and what did you learn from doing it?
\item What tools did you use to make your game? Please include software tools (e.g. programming languages or game engines) and any physical/analog materials (e.g. paper prototyping methods or storyboarding).
\end{list}


%\begin{table}
%\begin{tabular}{|p{\linewidth}|}
%\hline What was your initial goal for the game you made during the global game jam? \\
%\hline What inspirations or initial ideas did you have for your game? What was the starting inspirational source or goal for the game? \\
%\hline Why did you pick this particular idea for the game? \\
%\hline What problems did you encounter in developing your game? \\
%\hline What changes did you make to your initial idea as you worked on it during the game jam? Please describe the changes as small pieces of changes as possible. \\
%\hline What game mechanics and/or gameplay systems did you use in your game? \\
%\hline How did the mechanics or systems you made relate to the initial design ideas you had? \\
%\hline How did these mechanics change as you worked on the game during the game jam? \\
%\hline Did you prototype your game? If so, what kind of prototyping did you do and what did you learn from doing it? \\
%\hline What tools did you use to make your game? Please include software tools (e.g. programming languages or game engines) and any physical/analog materials (e.g. paper prototyping methods or storyboarding). \\
%\hline 
%\end{tabular}
%\caption{Survey questions.}
%\label{tab:survey} 
%\end{table}

\end{document}
